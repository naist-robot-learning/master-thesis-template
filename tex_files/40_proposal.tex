提案手法についてここに書く.

\vspace{10truemm}
%%%%% 論文書く上でのお役立ち情報 %%%%%
次のページに図と表の追加についてまとめる.
\newpage
図は以下のように追加する.ここで図は基本的にページ上部に配置する.
%%%%%%%%%%%%%%%%%
\begin{figure}[t]
    \centering
    \includegraphics[clip, width=0.3\linewidth]{figures/rll_logo.png}
    \caption{Our lab logo}
    \label{fig: our_logo}
\end{figure}
%%%%%%%%%%%%%%%%%

{\bf 図内の文字は全て英語}にするとともに{\bf 図題も英文で記載}する.
また,この図を参照したいときは\figref{fig: our_logo}とする.
文頭で参照する場合は以下のようにする.
\figureref{fig: our_logo}は文頭に参照されましたので,Fig.~ではなくFigure~になりました.\
図のサイズは\texttt{includegraphics[width=]}で調整できる.現状は横幅の$0.3$倍.

複数の図を追加したいときはこうする.
%%%%%%%%%%%%%%%%%
\begin{figure}[t]
    \centering
  \begin{minipage}{0.49\linewidth}
    \centering
    \includegraphics[clip, trim=0 0 0 0, width=0.5\linewidth]{figures/logomark_Rnasi.png}
    \subcaption{NAIST logo}
    \label{fig: naist_logo}
  \end{minipage}
  \begin{minipage}{0.49\linewidth}
    \centering
    \includegraphics[clip, trim=0 0 0 0, width=0.5\linewidth]{figures/rll_logo.png}
    \subcaption{Our lab logo}
    \label{fig: rll_logo}
  \end{minipage}
  \caption{Logo comparison}
  \label{fig: logo_comp}
\end{figure}
%%%%%%%%%%%%%%%%%

各図はこのように\figref{fig: naist_logo},\figref{fig: rll_logo}と参照できるし,図全体も\figref{fig: logo_comp}と参照できる.
ここでは,\texttt{minipage}でまず全体を約半分に切った後で,その内部で$0.5$倍された図を追加している.

図は縦に並べたり,もっと複雑に並べたりすることも可能なので調べる.

\newpage
写真以外の図 (例えばパワポで作った図)はPDFなどのベクター形式で出力して使う.
\figref{fig: png_vs_pdf}にpngとPDFの比較を貼る.
PDFは拡大しても画質が劣化しない.
ベクター形式での画像出力はmacの場合スムーズに行える. \\(図として保存 $>$ PDF)

%%%%%%%%%%%%%%%%%
\begin{figure}[t]
    \centering
  \begin{minipage}{0.3\linewidth}
    \centering
    \includegraphics[clip, trim=0 0 0 0, width=0.9\linewidth]{figures/hit.png}
    \subcaption{PNG}
    \label{fig: png_img}
  \end{minipage}
  \begin{minipage}{0.3\linewidth}
    \centering
    \includegraphics[clip, trim=0 0 0 0, width=0.5\linewidth]{figures/hit_converted.pdf}
    \subcaption{PDF}
    \label{fig: converted_pdf_img}
  \end{minipage}
  \caption{Comparison of PNG and PDF}
  \label{fig: png_vs_pdf}
\end{figure}
%%%%%%%%%%%%%%%%%
%%%%%%%%%%%%%%%%%
\begin{figure}[t]
    \centering
    \includegraphics[clip, width=0.3\linewidth]{figures/hit.pdf}
    \caption{PDF image when not converted to shape}
    \label{fig: unconverted_pdf}
\end{figure}
%%%%%%%%%%%%%%%%%


上記のようにパワーポイントのアイコンを用いた場合の注意点として,\textbf{そのままではPDF形式で出力しても意図したものにならない}点が挙げられる.
\figref{fig: unconverted_pdf}にそのまま出力した場合の図を示す.
\figref{fig: converted_pdf_img}と比較すると両者ともPDFであるが画質に大きな差異がある.
これを回避するためには,パワポ上でアイコンを右クリックして\textbf{図形に変換}をクリックしてアイコンをばらす必要がある.

\newpage

表はこんな感じで入れることができる.表についても基本的に上部に配置する.

%%%%%%%%%%%%%%%%%
\begin{table}[t]
    \centering
    \begin{tabular}{l|cc}
        \toprule
                                & \textbf{Task A} & \textbf{Task B}  \\
        \midrule
        Number of dense units   & \multicolumn{2}{c}{512, 512, 512} \\
        Activation function     & \multicolumn{2}{c}{Swish} \\
        Input dimension         & 14 & 9 \\
        Output dimension        & 8  & 7 \\
        Number of ensembles     & 5  & 5 \\
        \bottomrule
    \end{tabular}
    \caption{This is the table.}
    \label{tab: sample}
\end{table}
%%%%%%%%%%%%%%%%%

multicolumnを使うと,$512, 512, 512$やSwishのようにセルを結合できる.
\tabref{tab: sample}は横幅が若干狭いので,手動で調整したものを\tabref{tab: improved_sample}にしめす.

%%%%%%%%%%%%%%%%%
\begin{table}[t]
    \centering
    \begin{tabular}{p{50truemm}|wc{30truemm}wc{30truemm}}
        \toprule
                                & \textbf{Task A} & \textbf{Task B}  \\
        \midrule
        Number of dense units   & \multicolumn{2}{c}{512, 512, 512} \\
        Activation function     & \multicolumn{2}{c}{Swish} \\
        Input dimension         & 14 & 9 \\
        Output dimension        & 8  & 7 \\
        Number of ensembles     & 5  & 5 \\
        \bottomrule
    \end{tabular}
    \caption{This is a more sophisticated table.}
    \label{tab: improved_sample}
\end{table}
%%%%%%%%%%%%%%%%%

なお,\TeX の表はこのサイト\footnote{\url{https://www.tablesgenerator.com/}}で比較的容易に作ることができる.