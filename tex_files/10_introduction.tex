\subsection{背景}
研究背景をここに書く.

\vspace{10truemm}
%%%%% 論文書く上でのお役立ち情報 %%%%%
\TeX ファイルは\texttt{tex\_files/}内に収める.
ファイルが章の順番通りに並ぶよう,ファイル名の先頭に数字を入れている($\eg$\texttt{{\red 00\_}abstract.tex}).
章などを増やしたい時は,\texttt{65\_real\_experiment\_result.tex}のようにする.

図などのファイルは\texttt{figures/}に入れる.
%%%%%%%%%%%%%%%%%%%%%%%%%%%%%%%%%%%

\subsection{研究目的}
研究目的をここに書く.

\vspace{10truemm}
%%%%% 論文書く上でのお役立ち情報 %%%%%
添削時に変更点だけ注目してほしいときは{\red このように色を変えることもできる.}
また,{\blue 青色}も使ったりする.(蛍光ハイライトはバグることがあるのであまり使わない方がよい.)
%%%%%%%%%%%%%%%%%%%%%%%%%%%%%%%%%%%

\subsection{アプローチ}
アプローチをここに書く.
アプローチについてよくわからない場合は,助教等に聞く.

\subsection{論文構成}
以下の例をいじる.\vspace{10truemm}


本論文では,\secref{sec: intro}で背景と研究目的について述べた.
\secref{sec: rel_work}では提案法と関わりのある関連研究について紹介し,\secref{sec: prelim}では提案法のベースとなる手法Xについて,定式化の準備をおこなう.
さらに,\secref{sec: proposal}では提案手法について説明し,\secref{sec: sim_exp}および\secref{sec: real_exp}ではそれぞれシミュレーション環境および実機環境の実験についてまとめる.
最後に,\secref{sec: concl}では本論文のまとめと今後の課題について述べる.