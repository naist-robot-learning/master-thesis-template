\subsection{強化学習}
手法を理解するために必要な要素について説明する.

\vspace{10truemm}

%%%%% 論文書く上でのお役立ち情報 %%%%%
本文中に文字式などを入れたいときは$x=2$とする.間違ってもx=2としない.

メートルなどの単位を使うときは,10mとせず\SI{10}{m}とする.(スペースに注目)

ベクトルと白抜き文字はそれぞれ$\mathbf{X}$と$\mathbb{N}$とする.

argmin, argmaxは$\argmax_x f(x)$,$\argmin_x f(x)$とできるが,本文中では下付きの$x$の位置が微妙なので本文中では$\inlineargmax_x f(x)$とする.

$\ie$や$\eg$,$\etc$もつかえる.

しっかりと数式を入れるときはこうする.
\begin{equation}
    p(y=0) = \frac{\exp \left(\eta f({\bf w}^0)\right)}{\exp \left(\eta f({\bf w}^0)\right) + \exp \left(\eta f({\bf w}^1)\right)} 
    \label{equ: unc_user_model}
\end{equation}
上式を参照したいときは\equref{equ: unc_user_model}とする.

式と図,表などは必ずラベル付けする.ここではラベルとして,\texttt{equ:~unc\_user\_model}を与えている.

数式のラベルには\texttt{equ:~},図には\texttt{fig:~},表には\texttt{tab:~}を頭につけることでラベル名が分かりやすくなるとともにラベル同士の衝突も減る.
%%%%%%%%%%%%%%%%%%%%%%%%%%%%%%%%%%%

\subsection{Sim2Real}