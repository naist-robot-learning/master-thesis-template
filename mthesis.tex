%%%%%%%%%%%%%%%%%%%%%%%%%%%%%%%%%%%%%%%%%%%%%
%%%%%%%%%%%%%%%%%%%%%%%%%%%%%%%%%%%%%%%%%%%%%
% ロボットラーニング研究室向け修論テンプレート
%
% 2023.12.18    Ver. 1.1 公開版 (Kwon)
% 2023.12.21    Ver. 1.2 副査7人まで対応できるように変更 (Kwon)
% 2024.01.10    Ver. 1.3 図題と表題を英語に (Kwon),PDFの見出しの文字化け対応
%%%%%%%%%%%%%%%%%%%%%%%%%%%%%%%%%%%%%%%%%%%%%
%%%%%%%%%%%%%%%%%%%%%%%%%%%%%%%%%%%%%%%%%%%%%

\documentclass[12pt,dvipdfmx]{jarticle}
% \documentclass[a4paper,12pt]{jsreport}
%%%%%%%%%%%%%%%%%%%%%%%%%%%%%%%%%%%%%%%%%%%%%
% 必要に応じてパッケージを追加.読み込む順番に応じて挙動が変わることがあるので注意.
%%%%%%%%%%%%%%%%%%%%%%%%%%%%%%%%%%%%%%%%%%%%%
\usepackage[utf8]{inputenc}                                     % UTF-8対応
\usepackage{naist-jmthesis}                                     % ネットに落ちているものをいじったやつ
\usepackage{amsmath, amssymb}                                   % 数式関連
\usepackage{bm, bbm}                                            % 太字のベクトルと白抜き文字
\usepackage{siunitx}                                            % SI単位を正しく表記
\usepackage[dvipdfmx]{graphicx}                                 % 図関連
\usepackage[dvipdfmx]{color}                                    % 文字色を変えたい (教員に向け変更点などを強調したいときに使う)
\usepackage[font=footnotesize]{caption}                         % キャプション.(font=footnotesize)はIEEEフォーマットに寄せている.
\usepackage[font=footnotesize,subrefformat=parens]{subcaption}  % キャプション.(font=footnotesize)はIEEEフォーマットに寄せている.
\usepackage{enumerate}                                          % enumerateを拡張する
\usepackage{url}                                                % URL用
\usepackage{booktabs}                                           % 表を横罫線のみ表示するように
\usepackage{multirow}                                           % 表内のセルを結合できるように
\usepackage{algorithm, algorithmicx, algpseudocode}             % 疑似コード用.
\usepackage[nobreak]{cite}                                      % \cite{}を拡張する.
\usepackage[dvipdfmx]{hyperref}                                 % ハイパーリンク (\hypersetup{}内は設定)
\usepackage{pxjahyper}                                          % PDF見出しの文字化けを防ぐ
\hypersetup{setpagesize=false,
 bookmarksnumbered=true,%
 bookmarksopen=true,%
 colorlinks=true,%
 linkcolor=black,
 citecolor=black,}
%%%%%%%%%%%%%%%%%%%%%%%%%%%%%%%%%%%%%%%%%%%%%
% 新しい命令を定義.必要に応じて追加.
%%%%%%%%%%%%%%%%%%%%%%%%%%%%%%%%%%%%%%%%%%%%%
% refまわり
\newcommand{\figref}[1]{Fig.~\ref{#1}}
\newcommand{\figureref}[1]{Figure~\ref{#1}}
\newcommand{\tabref}[1]{Table~\ref{#1}}
\newcommand{\equref}[1]{式(\ref{#1})}
\newcommand{\algoref}[1]{Algorithm~\ref{#1}}
\newcommand{\secref}[1]{\ref{#1}章}
\newcommand{\appref}[1]{付録~\ref{#1}}
% 数式
\newcommand{\argmax}{\mathop{\rm arg~max}\limits}
\newcommand{\argmin}{\mathop{\rm arg~min}\limits}
\DeclareMathOperator*{\inlineargmax}{\rm arg~max}
\DeclareMathOperator*{\inlineargmin}{\rm arg~min}
\newcommand{\expected}{{\mathbb{E}}}
\newcommand{\trace}{{\mathrm{tr}}}
% 文字色変更 (赤と青だけ簡単に)
\newcommand{\red}{\color{red}}
\newcommand{\blue}{\color{blue}}
% イディオム
\newcommand{\etc}{\textit{etc.}}
\newcommand{\eg}{\textit{e.g.,}~}
\newcommand{\ie}{\textit{i.e.,}~}
\newcommand{\naive}{na\"ive }
\newcommand{\Naive}{Na\"ive }
\newcommand{\apriori}{\textit{a~priori~}}
%%%%%%%%%%%%%%%%%%%%%%%%%%%%%%%%%%%%%%%%%%%%%
%%%%%%%%%%%%%%%%%%%%%%%%%%%%%%%%%%%%%%%%%%%%%
% 修論情報
\pagestyle{final}           % Camera-Ready
\doctitle{\mastersthesis}   % 修士論文
\major{\engineering}        % \engineering OR \science (README.mdを参照)
\program{\ise}              % \ise OR \ds (README.mdを参照)
% タイトル
\title{ロボットラーニング研究室の\\修論テンプレート}
\etitle{Robot Learning Lab's Thesis Template\\for Master's Degree}
% 著者情報
\author{先端 太郎}
\eauthor{Tarou Sentan}
% 提出日
\esyear{2024}
\jsyear{令和6}
\smonth{1}
\sday{30}
% 指導教員 (8人まで対応)
\cmembers{松原 崇充}{(情報科学領域 教授)}
         {奈良 二郎}{(情報科学領域 教授)} % 副査1人目
         {生駒 三郎}{(情報科学領域 助教)} % 副査2人目
         {高山 四郎}{(情報科学領域 客員助教,大阪大学工学部 助教)} % 副査3人目
         {名無 五郎}{(情報科学領域 特任助教)} % 副査4人目
         {}{} % 副査5人目
         {}{} % 副査6人目
         {}{} % 副査7人目

% キーワード
\keywords{キーワード (4,5個)}
\ekeywords{Keywords (4-5 words)}
%%%%%%%%%%%%%%%%%%%%%%%%%%%%%%%%%%%%%%%%%%%%%
%%%%%%%%%%%%%%%%%%%%%%%%%%%%%%%%%%%%%%%%%%%%%
\begin{document}
\titlepage
\cmemberspage

%%%%%%%%%%%%%%%%%%%%%%%%%%%%%%%%%%%%%%%%%%%%%%%%%%%%%%%%%%%%%%%%%%%%%%%%%
% アブストラクト (和文,英文)
% 和文アブストラクト
\abstract{
和文のアブストラクトをここに記入する.
改行せずにすべて書く.
}

\eabstract{
Enter the English abstract here.
Do not break the line.
}

\firstabstract
\secondabstract

% 目次
\toc
\newpage
% 図と表の目次
\listoffigures
\listoftables
\newpage
\pagenumbering{arabic}

%%%%%%%%%%%%%%%%%%%%%%%%%%%%%%%%%%%%%%%%%%%%%%%%%%%%%%%%%%%%%%%%%%%%%%%%%
\section{序論} \label{sec: intro}
\subsection{背景}
研究背景をここに書く.

\vspace{10truemm}
%%%%% 論文書く上でのお役立ち情報 %%%%%
\TeX ファイルは\texttt{tex\_files/}内に収める.
ファイルが章の順番通りに並ぶよう,ファイル名の先頭に数字を入れている($\eg$\texttt{{\red 00\_}abstract.tex}).
章などを増やしたい時は,\texttt{65\_real\_experiment\_result.tex}のようにする.

図などのファイルは\texttt{figures/}に入れる.
%%%%%%%%%%%%%%%%%%%%%%%%%%%%%%%%%%%

\subsection{研究目的}
研究目的をここに書く.

\vspace{10truemm}
%%%%% 論文書く上でのお役立ち情報 %%%%%
添削時に変更点だけ注目してほしいときは{\red このように色を変えることもできる.}
また,{\blue 青色}も使ったりする.(蛍光ハイライトはバグることがあるのであまり使わない方がよい.)
%%%%%%%%%%%%%%%%%%%%%%%%%%%%%%%%%%%

\subsection{論文構成}
以下の例をいじる.\vspace{10truemm}


本論文では,\secref{sec: intro}で背景と研究目的について述べた.
\secref{sec: rel_work}では提案法と関わりのある関連研究について紹介し,\secref{sec: prelim}では提案法のベースとなる手法Xについて,定式化の準備をおこなう.
さらに,\secref{sec: proposal}では提案手法について説明し,\secref{sec: sim_exp}および\secref{sec: real_exp}ではそれぞれシミュレーション環境および実機環境の実験についてまとめる.
最後に,\secref{sec: concl}では本論文のまとめと今後の課題について述べる.
\newpage
\section{関連研究} \label{sec: rel_work}
\subsection{ロボットによる柔軟物体操作}
関連研究は関連する分野毎に分けて書く.

%%%%% 論文書く上でのお役立ち情報 %%%%%

\vspace{10truemm}

論文を引くときはこうする\cite{hinton2007learning}.
複数同時に引くときはこうする\cite{le2013building, ran2013genome, sutton2018reinforcement}.
修論で参照する論文は,\texttt{reference.bib}に集めておく.
集め方は
\begin{enumerate}
    \item Google Scholar\footnote{\url{https://scholar.google.co.jp/schhp?hl=ja}.脚注はこのように入れる}で論文を検索
    \item 見つけた論文の{\bf 引用}をクリック
    \item {\bf BibTeX}をクリック
    \item 表示された文字列をコピーして,\texttt{reference.bib}にペースト
\end{enumerate}
%%%%%%%%%%%%%%%%%%%%%%%%%%%%%%%%%%%

\subsection{sim2Real}

\newpage
\section{準備} \label{sec: prelim}
\subsection{強化学習}
手法を理解するために必要な要素について説明する.

\vspace{10truemm}

%%%%% 論文書く上でのお役立ち情報 %%%%%
本文中に文字式などを入れたいときは$x=2$とする.間違ってもx=2としない.

メートルなどの単位を使うときは,10mとせず\SI{10}{m}とする.(スペースに注目)

ベクトルと白抜き文字はそれぞれ$\mathbf{X}$と$\mathbb{N}$とする.

argmin, argmaxは$\argmax_x f(x)$,$\argmin_x f(x)$とできるが,本文中では下付きの$x$の位置が微妙なので本文中では$\inlineargmax_x f(x)$とする.

$\ie$や$\eg$,$\etc$もつかえる.

しっかりと数式を入れるときはこうする.
\begin{equation}
    p(y=0) = \frac{\exp \left(\eta f({\bf w}^0)\right)}{\exp \left(\eta f({\bf w}^0)\right) + \exp \left(\eta f({\bf w}^1)\right)} 
    \label{equ: unc_user_model}
\end{equation}
上式を参照したいときは\equref{equ: unc_user_model}とする.

式と図,表などは必ずラベル付けする.ここではラベルとして,\texttt{equ:~unc\_user\_model}を与えている.

数式のラベルには\texttt{equ:~},図には\texttt{fig:~},表には\texttt{tab:~}を頭につけることでラベル名が分かりやすくなるとともにラベル同士の衝突も減る.
%%%%%%%%%%%%%%%%%%%%%%%%%%%%%%%%%%%

\subsection{Sim2Real}
\newpage
\section{提案手法} \label{sec: proposal}
提案手法についてここに書く.

\vspace{10truemm}
%%%%% 論文書く上でのお役立ち情報 %%%%%
次のページに図と表の追加についてまとめる.
\newpage
図は以下のように追加する.ここで図は基本的にページ上部に配置する.
%%%%%%%%%%%%%%%%%
\begin{figure}[t]
    \centering
    \includegraphics[clip, width=0.3\linewidth]{figures/rll_logo.png}
    \caption{Our lab logo}
    \label{fig: our_logo}
\end{figure}
%%%%%%%%%%%%%%%%%

{\bf 図内の文字は全て英語}にするとともに{\bf 図題も英文で記載}する.
また,この図を参照したいときは\figref{fig: our_logo}とする.
文頭で参照する場合は以下のようにする.
\figureref{fig: our_logo}は文頭に参照されましたので,Fig.~ではなくFigure~になりました.\
図のサイズは\texttt{includegraphics[width=]}で調整できる.現状は横幅の$0.3$倍.

複数の図を追加したいときはこうする.
%%%%%%%%%%%%%%%%%
\begin{figure}[t]
    \centering
  \begin{minipage}{0.49\linewidth}
    \centering
    \includegraphics[clip, trim=0 0 0 0, width=0.5\linewidth]{figures/logomark_Rnasi.png}
    \subcaption{NAIST logo}
    \label{fig: naist_logo}
  \end{minipage}
  \begin{minipage}{0.49\linewidth}
    \centering
    \includegraphics[clip, trim=0 0 0 0, width=0.5\linewidth]{figures/rll_logo.png}
    \subcaption{Our lab logo}
    \label{fig: rll_logo}
  \end{minipage}
  \caption{Logo comparison}
  \label{fig: logo_comp}
\end{figure}
%%%%%%%%%%%%%%%%%

各図はこのように\figref{fig: naist_logo},\figref{fig: rll_logo}と参照できるし,図全体も\figref{fig: logo_comp}と参照できる.
ここでは,\texttt{minipage}でまず全体を約半分に切った後で,その内部で$0.5$倍された図を追加している.

図は縦に並べたり,もっと複雑に並べたりすることも可能なので調べる.

\newpage
写真以外の図 (例えばパワポで作った図)はPDFなどのベクター形式で出力して使う.
\figref{fig: png_vs_pdf}にpngとPDFの比較を貼る.
PDFは拡大しても画質が劣化しない.
ベクター形式での画像出力はmacの場合スムーズに行える. \\(図として保存 $>$ PDF)

%%%%%%%%%%%%%%%%%
\begin{figure}[t]
    \centering
  \begin{minipage}{0.3\linewidth}
    \centering
    \includegraphics[clip, trim=0 0 0 0, width=0.9\linewidth]{figures/hit.png}
    \subcaption{PNG}
    \label{fig: png_img}
  \end{minipage}
  \begin{minipage}{0.3\linewidth}
    \centering
    \includegraphics[clip, trim=0 0 0 0, width=0.5\linewidth]{figures/hit_converted.pdf}
    \subcaption{PDF}
    \label{fig: converted_pdf_img}
  \end{minipage}
  \caption{Comparison of PNG and PDF}
  \label{fig: png_vs_pdf}
\end{figure}
%%%%%%%%%%%%%%%%%
%%%%%%%%%%%%%%%%%
\begin{figure}[t]
    \centering
    \includegraphics[clip, width=0.3\linewidth]{figures/hit.pdf}
    \caption{PDF image when not converted to shape}
    \label{fig: unconverted_pdf}
\end{figure}
%%%%%%%%%%%%%%%%%


上記のようにパワーポイントのアイコンを用いた場合の注意点として,\textbf{そのままではPDF形式で出力しても意図したものにならない}点が挙げられる.
\figref{fig: unconverted_pdf}にそのまま出力した場合の図を示す.
\figref{fig: converted_pdf_img}と比較すると両者ともPDFであるが画質に大きな差異がある.
これを回避するためには,パワポ上でアイコンを右クリックして\textbf{図形に変換}をクリックしてアイコンをばらす必要がある.

\newpage

表はこんな感じで入れることができる.表についても基本的に上部に配置する.

%%%%%%%%%%%%%%%%%
\begin{table}[t]
    \centering
    \begin{tabular}{l|cc}
        \toprule
                                & \textbf{Task A} & \textbf{Task B}  \\
        \midrule
        Number of dense units   & \multicolumn{2}{c}{512, 512, 512} \\
        Activation function     & \multicolumn{2}{c}{Swish} \\
        Input dimension         & 14 & 9 \\
        Output dimension        & 8  & 7 \\
        Number of ensembles     & 5  & 5 \\
        \bottomrule
    \end{tabular}
    \caption{This is the table.}
    \label{tab: sample}
\end{table}
%%%%%%%%%%%%%%%%%

multicolumnを使うと,$512, 512, 512$やSwishのようにセルを結合できる.
\tabref{tab: sample}は横幅が若干狭いので,手動で調整したものを\tabref{tab: improved_sample}にしめす.

%%%%%%%%%%%%%%%%%
\begin{table}[t]
    \centering
    \begin{tabular}{p{50truemm}|wc{30truemm}wc{30truemm}}
        \toprule
                                & \textbf{Task A} & \textbf{Task B}  \\
        \midrule
        Number of dense units   & \multicolumn{2}{c}{512, 512, 512} \\
        Activation function     & \multicolumn{2}{c}{Swish} \\
        Input dimension         & 14 & 9 \\
        Output dimension        & 8  & 7 \\
        Number of ensembles     & 5  & 5 \\
        \bottomrule
    \end{tabular}
    \caption{This is a more sophisticated table.}
    \label{tab: improved_sample}
\end{table}
%%%%%%%%%%%%%%%%%

なお,\TeX の表はこのサイト\footnote{\url{https://www.tablesgenerator.com/}}で比較的容易に作ることができる.
\newpage
\section{シミュレーション実験} \label{sec: sim_exp}
\subsection{実験概要}
シミュレーション実験の概要 (目的など)をここに書く.

\subsection{実験設定}
シミュレーション実験の設定をここに書く.

\subsection{実験結果}
シミュレーション実験の結果をここに書く.

\newpage
\section{実機実験} \label{sec: real_exp}
\subsection{実験概要}
実機実験の概要 (目的など)をここに書く.

\subsection{実験設定}
実機実験の設定をここに書く.

\subsection{実験結果}
実機実験の結果をここに書く.
\newpage
% 結論
\section{結論} \label{sec: concl}
\subsection{まとめ}
本論文のまとめをここに書く.

\subsection{今後の課題}
以下の例をいじる.課題の数は3である必要なし.

\vspace{10truemm}
本研究の今後の課題として以下の三つが挙げられる.

\begin{itemize}
    \item {\bf 一つ目の課題をここに}: 一つ目の課題の詳細をここに
    \item {\bf 二つ目の課題をここに}: 二つ目の課題の詳細をここに
    \item {\bf 三つ目の課題をここに}: 三つ目の課題の詳細をここに
\end{itemize}
\newpage
% 謝辞
\acknowledgements
謝辞だけは誰でも読んで理解することが可能です.

慎重に記入して下さい.
\newpage

% 参考文献
\bibliographystyle{plain}
\bibliography{reference}

% 付録
\appendix
\section{ふろくその1}
何を付録に含めるべきかは指導教員と相談してください.

付録が不要なら,\texttt{mthesis.tex}の該当箇所をコメントアウトしてください.
%%%%%%%%%%%%%%%%%%%%%%%%%%%%%%%%%%%%%%%%%%%%%
\end{document}
