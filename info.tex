% 通常は修士論文を選択.教員からの指示があったときのみ課題研究を選択すること.
%       選択肢: \mastersthesis (修士論文),\mastersreport (課題研究) 
\doctitle{\mastersthesis}

% 授与される学位を選択.選定にあたっては指導教員と相談すること.
%       選択肢: \engineering (工学修士),\science (理学修士)
\major{\engineering}

% 自身の教育プログラムを選択.
%       選択肢: \ise (情報理工学プログラム),\ds (データサイエンスプログラム),\dgi (デジタルグリーンイノベーションプログラム)
% ※上記に当てはまらない場合は`naist-rll-jmthesis.sty`の該当箇所を編集する必要あり
\program{\dgi}

% 和文タイトル."\\"を使って適切に改行すること.
\title{ロボットラーニング研究室の\\修論テンプレート (Ver.\:\:1.6)}
% 英文タイトル."\\"を使って適切に改行すること.
\etitle{Robot Learning Lab's Thesis Template\\for Master's Degree}

% 著者情報 (和).姓と名の間に半角スペースを入れること.
\author{先端 太郎}
% 著者情報 (英)
\eauthor{Tarou Sentan}

% 提出年月日
\esyear{2024}
\jsyear{令和6}
\smonth{1}
\sday{30}

% 指導教員 (8人まで対応).
% 副査がいない or 少ない場合でも "{}{}" を消さないこと!
\cmembers{松原 崇充}{(情報科学領域 教授)}
         {奈良 二郎}{(情報科学領域 教授)} % 副査1人目
         {生駒 三郎}{(情報科学領域 助教)} % 副査2人目
         {高山 四郎}{(情報科学領域 客員助教,大阪大学工学部 助教)} % 副査3人目
         {名無 五郎}{(情報科学領域 特任助教)} % 副査4人目
         {}{} % 副査5人目
         {}{} % 副査6人目
         {}{} % 副査7人目

% 4,5個のキーワード (和).選定にあたっては指導教員と相談すること.
\keywords{キーワード (4,5個)}
% 4,5個のキーワード (英).選定にあたっては指導教員と相談すること.
\ekeywords{Keywords (4-5 words)}
